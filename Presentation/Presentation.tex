\documentclass{beamer}

% Theme settings (choose one, or customize)
\usetheme{default}    % Default theme
% \usetheme{Madrid}    % Madrid theme
% \usetheme{Warsaw}    % Warsaw theme
% ... (there are many other themes)

% Title page information
\title{Consumer Theory and Behavioural Analysis}
\subtitle{(SOS 2023)}
\author{Satvik Jain}
\date{\today}

\begin{document}

\begin{frame}
    \titlepage
\end{frame}



\begin{frame}
    \frametitle{Topics}
    \begin{itemize}
        \item Overview
        \item Rational Consumer Behaviour
        \item Behavioural Economics
    \end{itemize}
\end{frame}

\begin{frame}
    \frametitle{Consumer Theory Overview}
    \begin{block}{Definition}
        Consumer theory is a branch of microeconomics that studies how consumers make choices about what goods and services to purchase.
    \end{block}
    \begin{block}{Key Concepts}
Utility: The satisfaction or pleasure consumers derive from consuming a product.\\
Budget Constraint: The limitation on consumer choices due to income and prices of goods.\\
Preferences: Individual tastes and subjective judgments that influence consumer decisions.
    \end{block}
\end{frame}


\begin{frame}
    \frametitle{Rational Consumer Behavior}
    \begin{block}{Explaination}
         In classical consumer theory, consumers are assumed to be rational decision-makers.
    \end{block}
    \begin{block}{Rationality Assumptions}
Completeness: Consumers can rank their preferences for all available options.\\
Transitivity: If option A is preferred to option B and option B is preferred to option C, then A is preferred to C.\\
Non-Satiation: More is preferred to less (i.e., consumers always want more of a good).
    \end{block}
\end{frame}


\begin{frame}
    \frametitle{Behavioral Economics}
    \begin{block}{Explaination}
         A field that combines insights from psychology and economics to understand how real-world consumers make decisions.
    \end{block}
    \begin{block}{Departures from Rationality}
Bounded Rationality: Consumers may have limited cognitive abilities to make complex decisions.\\
Behavioral Biases: Common biases like loss aversion, present bias, and anchoring that influence decision-making.\\
Heuristics: Mental shortcuts that simplify decision-making but can lead to suboptimal choices.
    \end{block}
\end{frame}



\begin{frame}
    \frametitle{The End}
    Thank you for your attention.
\end{frame}

\end{document}
